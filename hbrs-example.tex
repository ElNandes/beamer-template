\documentclass[aspectratio=169]{beamer}


\usepackage[utf8]{inputenc}
\usepackage{amsmath}
\usepackage{amsfonts}
%\usepackage{biblatex}
\usepackage{amssymb}
\usepackage{xcolor,colortbl}
\usepackage{graphicx}
\usepackage{ragged2e}  % `\justifying` text
\usepackage{booktabs}  % Tables
\usepackage{tabularx}
\usepackage{tikz}      % Diagrams
\usetikzlibrary{calc, shapes, backgrounds}
\usepackage{amsmath}
\definecolor{Gray}{gray}{0.85}
\usepackage{amssymb}
\usepackage{subcaption}
\usepackage{dsfont}
%%\usepackage{url}       % `\url
\usepackage{listings}  % Code listings
\usepackage[T1]{fontenc}
\usepackage{filecontents}
\usepackage[style=verbose,backend=biber]{biblatex}
\addbibresource{bibliography.bib}
\font\myfont=cmr12 at 10pt


\usepackage{theme/beamerthemehbrs}

\author[Elanton Fernandes]{Elanton Fernandes}
\subtitle{ Active Learning Loop}
\title{Analysis of Active Learning Mechanism Applied to Language Models for Computer Assisted Short Answer Grading}
\institute[HBRS]{Hochschule Bonn-Rhein-Sieg}
\date{\today}
\subject{Test beamer}

% leave the value of this argument empty if the advisors
% should not be included on the title slide
\def\advisors{Prof. Dr. Paul G. Pl{\"o}ger, M.Sc Tim Metzler}

% \thirdpartylogo{path/to/your/image}


\begin{document}
{
\begin{frame}
\titlepage
\end{frame}
\begin{frame}{Agenda}
\tableofcontents
\end{frame}
}


\section{Motivation}
\begin{frame}{Motivation}
	In universities with an increase in number of student every semester, the number of tests conducted also increases. This means that:
	\begin{itemize}
		\item The professor spends more time in correcting student exams than preparing for lectures.
		\item If students are not assigned full scores for on a test, they expect a meaningful feedback from the professor.
	\end{itemize}
\end{frame}
\begin{frame}{Motivation}
	Consider the following dummy scenario:
	\begin{itemize}
		\item 80 students enrolled in a class.
		\item Tests are conducted bi-weekly.
		\item Professor requires 15 minutes to evaluate one student test.
		\item Total time spent by the professor to evaluate all tests per week is 10 hours. 
	\end{itemize}
\end{frame}
\section{Problem Statement}
\begin{frame}{Problem Statement}
	%\framesubtitle{Problem Statement}
	\begin{itemize}
		\item Automate the evaluation of student tests while still keeping the oracle/professor in the loop.
		\item Allow the assignment of meaningful feedback to student answers indicating their mistakes.
	\end{itemize}
	
\end{frame}

\section{State of the Art}
\begin{frame}{Related Work}
\begin{itemize}
	\item Wu et al. (2021) designed an system to assign feedback called ProtoTransformer for evaluating programming based questions but not for short text answers. It used limited number of examples.
	
	\item Ghavidel et al. (2020) passed raw text through a transformer as input and used the output of classification model (CLS) token as feature.
	
	\item Mieskes and Pado, (2018) compared score assignment between automated and human assignment for RF, SVM, and DT classifiers across multiple datasets.
\end{itemize}

\end{frame}
\section{Approach}

\begin{frame}{Approach}
\framesubtitle{Contributions}
\begin{itemize}
	\item Implement four methods to alter text for feature extraction.
	\item Implement feedback assignment for short text answers.
	\item Test performance with five pre-trained language models and two classifiers.
\end{itemize}
\end{frame}
\section{Approach}
\begin{frame}{Approach}
	\framesubtitle{Training Cycle}
	\begin{figure}
		\centering
		\includegraphics[scale = 0.67]{images/architecture_training.pdf}
		\label{fig:architecture train}
	\end{figure}
\end{frame}
\begin{frame}{Approach}
\framesubtitle{Prediction Cycle}
\begin{figure}
	\centering
	\includegraphics[scale = 0.75]{images/architecture_prediction.pdf}
	\label{fig:architecture predict}
\end{figure}
\end{frame}

\begin{frame}{Dataset}
\begin{table}
	\begin{tabular}{|c|c|c|c|}
		\hline
		 Dataset & Domain & No. of Questions Pairs & No. of Responses\\
		\hline 
		Mohler \footfullcite{mohler-etal-2011-learning}& Computer Science& 81 & 2237 \\
		\hline
		NN Exam & Neural Network \& AI & 40 & 1137 \\
		\hline
		AMR Exam & Robotics & 5 & 190 \\
		\hline
	\end{tabular}
	\caption{}
\end{table}
\end{frame}
\begin{frame}{Approach}
\framesubtitle{Uncertainty Sampling}
Uncertainty sampling is a query strategy that queries the instances about which it is least certain how to label. We use uncertainty sampling variant might query the instance whose prediction is the least confident:
\begin{equation}
\label{equation:uncertainty sampling}
\scalebox{1.2}{$x_{LC} = argmin_{x} P({\hat{y}|x;\theta})$}
\end{equation}
Where $x$ is the feature, $y$ is the class label prediction, and $\hat{y} = argmax_y P({y|x;\theta})$ is the class label that has the largest posterior probability using model $\theta$.
\end{frame}
\begin{frame}{Approach}
\framesubtitle{Feature Extraction: Overview}
\begin{figure}
	\centering
	\includegraphics[scale = 0.65]{images/feature_extraction.pdf}
	\label{fig:feature extraction}
\end{figure}
\end{frame}
\begin{frame}{Approach}
\framesubtitle{Feature Extraction: Passage-Based Method}
\begin{figure}
	\centering
	\includegraphics[scale = 1]{images/passage_FE_slides.pdf}
	\label{fig:passage fe slides}
\end{figure}
\end{frame}
\begin{frame}{Approach}
\framesubtitle{Feature Extraction: Sentence-Based Method}
\begin{figure}
	\centering
	\includegraphics[scale = 1]{images/sentence_FE_slides.pdf}
	\label{fig:sentence fe slides}
\end{figure}
\end{frame}
\begin{frame}{Approach}
\framesubtitle{Feature Extraction: Chunk-Based Method}
\begin{figure}
	\centering
	\includegraphics[scale = 0.6]{images/chunk_FE_slides.pdf}
	\label{fig:chunk fe}
\end{figure}
\end{frame}
\begin{frame}{Approach}
\framesubtitle{Feature Extraction: Resource Description Framework (RDF) Based Method}
\begin{figure}
	\centering
	\includegraphics[scale = 1]{images/RDF_FE_slides.pdf}
	\label{fig:rdf fe}
\end{figure}
\end{frame}
\begin{frame}{Approach}
\framesubtitle{Language Models}
\begin{table}
	\begin{center}
		\begin{tabular}{ |c|c|c| }
			\hline
			Language Model: & Base model & Number \\&&Training tuples  
			\\ \hline 
			all-mpnet-base-v2\cite{SBERT} & 	microsoft/mpnet-base. &1.17B
			
			\\ \hline
			all-distilroberta-v1\cite{SBERT} & 	distilroberta-base &1.12B
			\\ \hline
			all-MiniLM-L12-v2\cite{SBERT} & 		microsoft/MiniLM-L12-H384-uncased &1.17B
			\\ \hline
			multi-qa-distilbert-cos-v1\cite{SBERT} & 	distilbert-base &214M
			\\ \hline
			all-MiniLM-L6-v2\cite{SBERT} & 		nreimers/MiniLM-L6-H384-uncased &1.17B
			\\ \hline
		\end{tabular}
		\caption{Displays pre-trained language models with their base model used in training and number of training tuples used\cite{SBERT}.}
		\label{table:language models}
	\end{center}
\end{table}
\end{frame}
\section{Evaluation}
\begin{frame}{Evaluation}
	\framesubtitle{Score}
	\begin{itemize}
		\item Pearsons Correlation
		\begin{equation}
		\label{equation:pearson correlation}
		\rho(y,\hat{y}) = \frac{cov(\vec{y},\hat{\vec{y}})}{\sigma_y \sigma_{\hat{y}}}
		\end{equation}
		\item RMSE Score
		\begin{equation}
		\label{equation:Root mean square error}
		RMSE = \sqrt{\frac{1}{n}\Sigma_{i=1}^n(\hat{\vec{y_i}} - \vec{y}_i)^2}
		\end{equation}
	\end{itemize}
Where $\vec{y}$ represents actual grade and $\hat{\vec{y}}$ represents predicted grade with $\sigma_y$ and $\sigma_{\hat{y}}$ computed as the standard deviation of $\vec{y}$ and $\hat{\vec{y}}$
\end{frame}
\begin{frame}{Evaluation}
\framesubtitle{Feedback}
\begin{table}
	\begin{center}
		\begin{tabular}{ |c|c| }
			\hline
			Question & What is a variable?  
			\\ \hline 
			Reference Answer & A location in memory that can store a value.
			\\ \hline
			Student Answer & A value/word that can assume any of a set of values
			\\ \hline
			Feedback A & Correct
			\\ \hline
			Feedback B & Missing keywords: Location in memory
			\\ \hline
			Feedback C & A variable is a location in memory that stores a value
			\\ \hline
		\end{tabular}
		\caption{Presented survey to participants.}
		\label{table:language models}
	\end{center}
\end{table}
\centering
$Agreement\ Score$ $= \frac{Model\ generated\ most\ rated\ feedback}{Total\ Number\ of\ Participants}$
\end{frame}
\section{Results}
\begin{frame}{Results}
	\framesubtitle{Notations}
	\begin{columns}
		\begin{column}{0.5\textwidth}
			\begin{table}
				\centering
				\begin{tabular}{|c|c|}
					\hline
					\rowcolor{Gray}
					Method & Notation\\
					\hline
					Passage-based Methods & M1\\
					\hline
					Sentence-based Method & M2\\
					\hline
					Chunk-based Method & M3\\
					\hline
					RDF-based Method & M4\\
					\hline
				\end{tabular}
			\end{table}
		\end{column}
	\begin{column}{0.5\textwidth}
		\begin{table}
			\centering
			\begin{tabular}{|c|c|}
				\hline
				\rowcolor{Gray}
				Language Model & Notation\\
				\hline
				all-mpnet-base-v2 & LM1\\
				\hline
				all-distilroberta-v1 & LM2\\
				\hline
				all-MiniLM-L12-v2 & LM3\\
				\hline
				multi-qa-distilbert-cos-v1 & LM4\\
				\hline
				all-MiniLM-L6-v2 & LM5\\
				\hline
				
			\end{tabular}
		\end{table}
	\end{column}
	\end{columns}
	
\end{frame}
\begin{frame}{Results}
	\framesubtitle{Score: Pearson Correlation (Methods)}
\begin{table}
	\centering
	\begin{subtable}[c]{0.5\textwidth}
		
		\begin{tabular}{|c|c|c|c|c|}
			\hline
			Dataset & M1 & M2 & M3 & M4 \\
			\hline
			Mohler  & \underline{\textbf{0.826}}  &\textbf{0.791} &\textbf{0.816} &0.782 \\
			\hline
			NN Exam  &\textbf{0.941} &\textbf{0.828} &0.561 &\textbf{0.846} \\
			\hline
			AMR Exam  &\textbf{0.658} &0.458 &\textbf{0.640} & \textbf{0.428} \\
			\hline
		\end{tabular}
		\subcaption{}
	\end{subtable}
\centering
	\begin{subtable}[c]{0.5\textwidth}
		
		\begin{tabular}{|c|c|c|c|c|}
			\hline
			Dataset & M1 & M2 & M3 & M4 \\
			\hline
			Mohler &0.689  &0.627 &0.687 &\textbf{0.792} \\
			\hline
			NN Exam &0.889 &0.791 &\textbf{0.638} &0.664 \\
			\hline
			AMR Exam &0.622 &\textbf{0.474} &0.593 &\textbf{0.428} \\
			\hline
		\end{tabular}	
		\subcaption{}
	\end{subtable}
	\caption{Comparison of Pearson Correlation between Random Forest (a) and AdaBoost (b) classifiers. Where M1: Passage-based, M2: Sentence-based, M3:Chunk-based, and M4: RDF-based method.}
\end{table}
\end{frame}

\begin{frame}{Results}
\framesubtitle{Score: Pearson Correlation (Language Models)}
\noindent
\begin{table}
	\noindent
	\begin{subtable}[c]{0.7\textwidth}
		\noindent
		\centering
		\begin{tabular}{|c|c|c|c|c|c|}
			\hline
			Dataset & LM1 & LM2 & LM3 & LM4 & LM5 \\
			\hline
			Mohler &\textbf{0.802}& \textbf{0.797}& \textbf{0.796}& \textbf{0.796}& \textbf{0.789}\\
			\hline
			NN Exam &\textbf{0.732}& \textbf{0.670}& \textbf{0.705}& \textbf{0.755}& \textbf{0.760}\\
			\hline
			AMR Exam &0.453& \textbf{0.518}& \textbf{0.525}& \textbf{0.523}& \textbf{0.503}\\
			\hline
		\end{tabular}
		\subcaption{}
	\end{subtable}
	\begin{subtable}[c]{0.7\textwidth}
		\centering
		\begin{tabular}{|c|c|c|c|c|c|}
			\hline
			Dataset & LM1 & LM2 & LM3 & LM4 & LM5 \\
			\hline
			Mohler &0.659& 0.673& 0.211& 0.544& 0.499 \\
			\hline
			NN Exam &0.614& 0.653& 0.704& 0.698& 0.605\\
			\hline
			AMR Exam &\textbf{0.502}& 0.440& 0.430& 0.508& 0.467\\
			\hline
		\end{tabular}	
		\subcaption{}
	\end{subtable}
	\caption{Comparison of Pearson Correlation between Random Forest (a) and AdaBoost (b) classifiers with language models (LM).}
\end{table}
\end{frame}


%Rmse
\begin{frame}{Results}
\framesubtitle{Score: Root Mean Square Error (Methods)}
\begin{table}
	\begin{subtable}[c]{0.5\textwidth}
		\centering
		\begin{tabular}{|c|c|c|c|c|}
			\hline
			Dataset & M1 & M2 & M3 & M4 \\
			\hline
			Mohler & \textbf{0.893}  &\textbf{0.949} &\textbf{0.920} &0.942 \\
			\hline
			NN Exam &\textbf{0.296} &\textbf{0.520} &\textbf{0.433} &\textbf{0.522} \\
			\hline
			AMR Exam &\textbf{0.596} &0.716 &\textbf{0.596} & \textbf{0.736} \\
			\hline
		\end{tabular}
		\subcaption{}
	\end{subtable}
	\begin{subtable}[c]{0.5\textwidth}
		\centering
		\begin{tabular}{|c|c|c|c|c|}
			\hline
			Dataset & M1 & M2 & M3 & M4 \\
			\hline
			Mohler &1.218  &1.226 &1.169 &\textbf{0.920} \\
			\hline
			NN Exam &0.405 &0.571 &{0.495} &0.741 \\
			\hline
			AMR Exam &0.616 &\textbf{0.707} &0.630 &{0.741} \\
			\hline
		\end{tabular}	
		\subcaption{}
	\end{subtable}
	\caption{Comparison of RMSE score between Random Forest (a) and AdaBoost (b) classifiers with methods (M).}
\end{table}
\end{frame}

\begin{frame}{Results}
\framesubtitle{Score: Root Mean Square Error (Language Models)}
\begin{table}
\begin{subtable}[c]{0.5\textwidth}
	\centering
	\begin{tabular}{|c|c|c|c|c|c|}
		\hline
		Dataset & LM1 & LM2 & LM3 & LM4 & LM5 \\
		\hline
		Mohler &\textbf{0.931}& \textbf{0.941}& \textbf{0.941}& \textbf{0.941}& \textbf{0.956}\\
		\hline
		NN Exam &\textbf{0.484}& {0.591}& \textbf{0.558}& \textbf{0.490}& \textbf{0.492}\\
		\hline
		AMR Exam &0.735& \textbf{0.680}& \textbf{0.676}& 0.684& \textbf{0.698}\\
		\hline
	\end{tabular}
	\subcaption{}
\end{subtable}
\begin{subtable}[c]{0.5\textwidth}
	\centering
	\begin{tabular}{|c|c|c|c|c|c|}
		\hline
		Dataset & LM1 & LM2 & LM3 & LM4 & LM5 \\
		\hline
		Mohler &1.182& 1.163& 1.667& 1.278& 1.363 \\
		\hline
		NN Exam &0.632& \textbf{0.582}& 0.587& 0.587& 0.650\\
		\hline
		AMR Exam &\textbf{0.692}& 0.748& 0.718& \textbf{0.682}& 0.736\\
		\hline
	\end{tabular}	
	\subcaption{}
\end{subtable}
\caption{Comparison of RMSE score between Random Forest (a) and AdaBoost (b) classifiers with language models (LM).}
\end{table}
\end{frame}
%Feedbacks
\begin{frame}{Results}
	\framesubtitle{Feedback: Survey Results}
	\begin{columns}
		\begin{column}{0.65\textwidth}
		\begin{table}
			\begin{tabular}{ |c|c| }
				\hline
				Question & What is a variable?  
				\\ \hline 
				Reference Answer & A location in memory\\& that can store a value.
				\\ \hline
				Student Answer & a value/word that can\\& assume any of a set of values
				\\ \hline
				Feedback A & correct
				\\ \hline
				Feedback B & missing keywords:\\& location in memory
				\\ \hline
				Feedback C & A variable is a location\\& in memory that stores a value
				\\ \hline
			\end{tabular}
		\end{table}
	\end{column}
	\begin{column}{0.5\textwidth}
		\begin{center}
			\includegraphics[width=1\textwidth]{images/survey_1.pdf}
		\end{center}
	\end{column}
	\end{columns}
\end{frame}
\begin{frame}{Results}
	\framesubtitle{Feedback: Agreement Scores (Methods)}
	\begin{columns}
		\begin{column}{1\textwidth}
	\begin{table}
			\centering
			\begin{tabular}{|c|c|c|c|c|}
				
				\hline
				\rowcolor{Gray}
				Classifier&\multicolumn{4}{|c|}{Methods}\\
				\hline
				 & M1 & M2 & M3 & M4 \\
				\hline
				Random Forest &\textbf{60.00}&22.73&31.82&35.91\\
				\hline
				AdaBoost &\textbf{60.00}&22.73&31.82&35.91\\
				\hline
			\end{tabular}
		
		\caption{Mean agreement scores for Random Forest (a) and AdaBoost Classifier (b) with methods.}
	\end{table}
\end{column}
\end{columns}
\end{frame}
\begin{frame}{Results}
	\framesubtitle{Feedback: Agreement Scores (Models)}
	\begin{table}
			\centering
			\begin{tabular}{|c|c|c|c|c|c|}
			\hline
			Classifier & LM1 & LM2 & LM3 & LM4 & LM5 \\
			\hline
			Random Forest &25.11& 26.82& 24.66& \textbf{37.05}& 21.25\\
			\hline
			AdaBoost &25.11& 26.82& 24.66& \textbf{37.05}& 21.25\\
			\hline
			\end{tabular}
		\caption{Mean agreement scores for Random Forest and AdaBoost Classifier with Language Models.}
	\end{table}
	
\end{frame}
\begin{frame}{Results}
	\framesubtitle{Summary: Scores}
	\begin{columns}
		\begin{column}{0.5\textwidth}
			\begin{table}
				\centering
				\begin{tabular}{|c|c|c|c|}
					\hline
					Dataset & Method & Model & CL \\
					\hline
					Mohler & M1 & LM1& RF \\
					\hline
					NN Exam & M1 & LM5 & RF\\
					\hline
					AMR Exam & M1& LM3 & RF\\
					\hline	
				\end{tabular}
			\caption{Pearson Correlation Performance Summary}
			\end{table}
		\end{column}
		\begin{column}{0.5\textwidth}
			\begin{table}
			\centering
			\begin{tabular}{|c|c|c|c|}
				\hline
				Dataset & Method & Model & CL \\
				\hline
				Mohler & M3 & LM1\& LM4& RF \\
				\hline
				NN Exam & M1 & LM2 & RF\\
				\hline
				AMR Exam & M1\& M3 & LM3 & RF\\
				\hline	
			\end{tabular}
			\caption{RMSE Score Performance Summary}
		\end{table}
		\end{column}
	\end{columns}
\end{frame}
\begin{frame}{Results}
	\framesubtitle{Feedback}
	\begin{table}
		\centering
		\begin{tabular}{|c|c|c|c|c|}
			\hline
			Dataset & Method & Model & Method-Model & Classifier \\
			\hline
			Mohler & M1 & LM4 & M1-LM4 & RF \\
			\hline
		\end{tabular}
	\caption{Results of feedback evaluation}
	\end{table}
\end{frame}
\section{Summary}
\begin{frame}{Summary}
In this project the following was done:
	\begin{itemize}
		
		\item Four methods were implemented to alter student answer text.
		\item Pearson Correlation and RMSE score were used as metrics for score evaluation.
		\item A survey was created and used in the evaluation of the feedback assigned by the model.
	\end{itemize}
\end{frame}
\section{Future Work}



%    \begin{frame}[label=bibliography]{References}
%      \begin{thebibliography}{9}
%      	\bibitem{SBERT}
%      		Reimers, Nils and Gurevych, Iryna ``Sentence-BERT: Sentence Embeddings using Siamese BERT-Networks'', 2019.
%      		
%        \bibitem{mohler}
%            M. Mohler, R. Bunescu, and R. Mihalcea, “Learning to grade short answer questions using semantic
%            similarity measures and dependency graph alignments,” in Proceedings of the 49th Annual Meeting of
%            the Association for Computational Linguistics: Human Language Technologies, (Portland, Oregon,
%            USA), pp. 752–762, Association for Computational Linguistics, June 2011.
%        \bibitem{lamport94}
%            Leslie~Lamport.
%            \emph{\LaTeX : A Document Preparation System}.
%            Addison-Wesley, 1986.
%        \bibitem{MG94}
%            M.~Goossens, F.~Mittelbach, and A.~Samarin.
%            \emph{The \LaTeX\ Companion}.
%            Addison-Wesley, 1994.
%        \bibitem{tantau04}
%            Till~Tantau.
%            \emph{User's Guide to the Beamer Class Version 3.01}.
%            Available at \url{http://latex-beamer.sourceforge.net}.
%        \bibitem{MS05}
%            A.~Mertz and W.~Slough.
%            Edited by B.~Beeton and K.~Berry.
%            \emph{Beamer by example} In TUGboat,
%              Vol. 26, No. 1., pp. 68-73.
%      \end{thebibliography}
%    \end{frame}

\begin{frame}{Approach:Extra Slides}
\framesubtitle{Feature Extraction: Passage-based method}
\begin{figure}
\centering
\includegraphics[scale = 0.65]{images/passage_FE.pdf}
\label{fig:passage fe}
\end{figure}
\end{frame}
\begin{frame}{Approach}
\framesubtitle{Feature Extraction: Sentence-based method}
\begin{figure}
\centering
\includegraphics[scale = 0.65]{images/sentence_FE.pdf}
\label{fig:sentence fe}
\end{figure}
\end{frame}
\begin{frame}{Approach}
\framesubtitle{Feature Extraction: Chunk-based method}
\begin{figure}
\centering
\includegraphics[scale = 0.5]{images/chunk_FE.pdf}
\label{fig:chunk fe}
\end{figure}
\end{frame}
\begin{frame}{Approach}
\framesubtitle{Feature Extraction: RDF-based method}
\begin{figure}
\centering
\includegraphics[scale = 0.65]{images/RDF_FE.pdf}
\label{fig:rdf fe}
\end{figure}
\end{frame}

%\begin{frame}{Jabberwocky}
%      \framesubtitle{Lewis Carroll}%
%      \begin{tikzpicture}[overlay,remember picture]
%        \node[anchor=south east,xshift=-30pt,yshift=35pt]
%          at (current page.south east) {
%            %\includegraphics[width=35mm]{resources/jabberwocky-light}
%          };
%      \end{tikzpicture}%
%      'Twas brillig, and the slithy toves\\
%      Did gyre and gimble in the wabe;\\
%      All mimsy were the borogoves,\\
%      And the mome raths outgrabe.\\\bigskip
%
%      “Beware the Jabberwock, my son!\\
%      The jaws that bite, the claws that catch!\\
%      Beware the Jubjub bird, and shun\\
%      The frumious Bandersnatch!”\\
%\end{frame}
%
%
%\begin{frame}[label=lists]{Lists and locales}
%      \framesubtitle{Lorem ipsum dolor sit amet}
%      \begin{columns}[onlytextwidth]
%        \column{.5\textwidth}
%          \begin{itemize}
%            \item Nulla nec lacinia odio. Curabitur urna tellus.
%            \begin{itemize}
%              \item Fusce id sodales dolor. Sed id metus dui.
%              \begin{itemize}
%                \item Cupio virtus licet mi vel feugiat.
%              \end{itemize}
%            \end{itemize}
%          \end{itemize}
%        \column{.5\textwidth}
%          \begin{enumerate}
%            \item Donec porta, risus porttitor egestas scelerisque video.
%            \begin{enumerate}
%              \item Nunc non ante fringilla, manus potentis cario.
%              \begin{enumerate}
%                \item Pellentesque servus morbi tristique.
%              \end{enumerate}
%            \end{enumerate}
%          \end{enumerate}
%      \end{columns}
%      \bigskip
%      \justifying
%
%      {\uselanguage{czech}Nechť již hříšné saxofony ďáblů
%      rozzvučí síň úděsnými tóny waltzu, tanga a quickstepu!}
%      {\uselanguage{slovak} Nezvyčajné kŕdle šťastných figliarskych
%      ďatľov učia pri kótovanom ústí Váhu mĺkveho koňa Waldemara
%      obžierať väč\-šie kusy exkluzívnej kôry.}
%      {\uselanguage{english}The quick, brown fox jumps over a lazy
%      dog. DJs flock by when MTV ax quiz prog. “Now fax quiz Jack!”}
%\end{frame}
%
%\subsection{Structuring Elements}
%    \begin{frame}[label=simmonshall]{Text blocks}
%      \framesubtitle{In plain, example, and \alert{alert} flavour}
%      \alert{This text} is highlighted.
%
%      \begin{block}{A plain block}
%        This is a plain block containing some \alert{highlighted text}.
%      \end{block}
%      \begin{exampleblock}{An example block}
%        This is an example block containing some \alert{highlighted text}.
%      \end{exampleblock}
%      \begin{alertblock}{An alert block}
%        This is an alert block containing some \alert{highlighted text}.
%      \end{alertblock}
%    \end{frame}
%
%
%\begin{frame}[label=proof]{Definitions, theorems, and proofs}
%      \framesubtitle{All integers divide zero}
%      \begin{definition}
%        $\forall a,b\in\mathds{Z}: a\mid b\iff\exists c\in\mathds{Z}:a\cdot c=b$
%      \end{definition}
%      \begin{theorem}
%        $\forall a\in\mathds{Z}: a\mid 0$
%      \end{theorem}
%      \begin{proof}[Proof\nopunct]
%      	$\forall a\in \mathds{Z}:\cdot 0=0$
%      \end{proof}
%\end{frame}
%
%    \subsection{Numerals and Mathematics}
%    \begin{frame}[label=math]{Numerals and Mathematics}
%      \framesubtitle{Formulae, equations, and expressions}
%      \begin{columns}[onlytextwidth]
%        \column{.20\textwidth}
%          1234567890
%        \column{.20\textwidth}
%          \oldstylenums{1234567890}
%        \column{.20\textwidth}
%          $\hat{x}$, $\check{x}$, $\tilde{a}$,
%         $\bar{a}$, $\dot{y}$, $\ddot{y}$
%        \column{.40\textwidth}
%          $\int \!\! \int f(x,y,z)\,\mathsf{d}x\mathsf{d}y\mathsf{d}z$
%      \end{columns}
%      \begin{columns}[onlytextwidth]
%        \column{.5\textwidth}
%          $$\frac{1}{\displaystyle 1+
%            \frac{1}{\displaystyle 2+
%            \frac{1}{\displaystyle 3+x}}} +
%            \frac{1}{1+\frac{1}{2+\frac{1}{3+x}}}$$
%        \column{.5\textwidth}
%          $$F:\left| \begin{array}{ccc}
%          F''_{xx} & F''_{xy} &  F'_x \\
%          F''_{yx} & F''_{yy} &  F'_y \\
%          F'_x     & F'_y     & 0
%         \end{array}\right| = 0$$
%      \end{columns}
%      \begin{columns}[onlytextwidth]
%        \column{.3\textwidth}
%          $$\mathop{\int \!\!\! \int}_{\mathbf{x} \in \mathds{R}^2}
%          \! \langle \mathbf{x},\mathbf{y}\rangle\,\mathsf{d}\mathbf{x}$$
%        \column{.33\textwidth}
%         $$\overline{\overline{a\alpha}^2+\underline{b\beta}
%           +\overline{\overline{d\delta}}}$$
%        \column{.37\textwidth}
%          $\left] 0,1\right[ + \lceil x \rfloor - \langle x,y\rangle$
%      \end{columns}
%      \begin{columns}[onlytextwidth]
%        \column{.4\textwidth}
%          \begin{eqnarray*}
%           e^x &\approx& 1+x+x^2/2! + \\
%             && {}+x^3/3! + x^4/4!
%          \end{eqnarray*}
%        \column{.6\textwidth}
%          $${n+1\choose k} = {n\choose k} + {n \choose k-1}$$
%      \end{columns}
%    \end{frame}
%
%    \subsection{Figures and Code Listings}
%    \begin{frame}[label=figs1]{Figures}
%      \framesubtitle{Tables, graphs, and images}
%      \begin{table}[!b]
%%        {\carlitoTLF % Use monospaced lining figures
%        \begin{tabularx}{\textwidth}{Xrrr}
%          \textbf{Faculty} & \textbf{With \TeX} & \textbf{Total} &
%          \textbf{\%} \\
%          \toprule
%          Faculty of Informatics       & 1\,716  & 2\,904  &
%          59.09 \\% 1433
%          Faculty of Science           & 786     & 5\,275  &
%          14.90 \\% 1431
%          Faculty of $\genfrac{}{}{0pt}{}{\textsf{Economics and}}{%
%          \textsf{Administration}}$    & 64      & 4\,591  &
%          1.39  \\% 1456
%          Faculty of Arts              & 69      & 10\,000 &
%          0.69  \\% 1421
%          Faculty of Medicine          & 8       & 2\,014  &
%          0.40  \\% 1411
%          Faculty of Law               & 15      & 4\,824  &
%          0.31  \\% 1422
%          Faculty of Education         & 19      & 8\,219  &
%          0.23  \\% 1441
%          Faculty of Social Studies    & 12      & 5\,599  &
%          0.21  \\% 1423
%          Faculty of Sports Studies    & 3       & 2\,062  &
%          0.15  \\% 1451
%          \bottomrule
%        \end{tabularx}%}
%        \caption{The distribution of theses written using \TeX\ during 2010--15 at MU}
%      \end{table}
%    \end{frame}
%    \begin{frame}[label=figs2]{Figures}
%      \framesubtitle{Tables, graphs, and images}
%      \begin{figure}[b]
%        \centering
%        % Flipping a coin
%        % Author: cis
%        \tikzset{
%          head/.style = {fill = none, label = center:\textsf{H}},
%          tail/.style = {fill = none, label = center:\textsf{T}}}
%        \scalebox{0.65}{\begin{tikzpicture}[
%            scale = 1.5, transform shape, thick,
%            every node/.style = {draw, circle, minimum size = 10mm},
%            grow = down,  % alignment of characters
%            level 1/.style = {sibling distance=3cm},
%            level 2/.style = {sibling distance=4cm},
%            level 3/.style = {sibling distance=2cm},
%            level distance = 1.25cm
%          ]
%          \node[shape = rectangle,
%            minimum width = 6cm, font = \sffamily] {Coin flipping}
%          child { node[shape = circle split, draw, line width = 1pt,
%                  minimum size = 10mm, inner sep = 0mm, rotate = 30] (Start)
%                  { \rotatebox{-30}{H} \nodepart{lower} \rotatebox{-30}{T}}
%           child {   node [head] (A) {}
%             child { node [head] (B) {}}
%             child { node [tail] (C) {}}
%           }
%           child {   node [tail] (D) {}
%             child { node [head] (E) {}}
%             child { node [tail] (F) {}}
%           }
%          };
%
%          % Filling the root (Start)
%          \begin{scope}[on background layer, rotate=30]
%            \fill[head] (Start.base) ([xshift = 0mm]Start.east) arc (0:180:5mm)
%              -- cycle;
%            \fill[tail] (Start.base) ([xshift = 0pt]Start.west) arc (180:360:5mm)
%              -- cycle;
%          \end{scope}
%
%          % Labels
%          \begin{scope}[nodes = {draw = none}]
%            \path (Start) -- (A) node [near start, left]  {$0.5$};
%            \path (A)     -- (B) node [near start, left]  {$0.5$};
%            \path (A)     -- (C) node [near start, right] {$0.5$};
%            \path (Start) -- (D) node [near start, right] {$0.5$};
%            \path (D)     -- (E) node [near start, left]  {$0.5$};
%            \path (D)     -- (F) node [near start, right] {$0.5$};
%            \begin{scope}[nodes = {below = 11pt}]
%              \node [name = X] at (B) {$0.25$};
%              \node            at (C) {$0.25$};
%              \node [name = Y] at (E) {$0.25$};
%              \node            at (F) {$0.25$};
%            \end{scope}
%          \end{scope}
%        \end{tikzpicture}}
%        \caption{Tree of probabilities -- Flipping a coin\footnote[frame]{%
%          A derivative of a diagram from \url{texample.net} by cis, CC BY 2.5 licensed}}
%      \end{figure}
%    \end{frame}
%
%    \defverbatim[colored]\sleepSort{
%      \begin{lstlisting}[language=C,tabsize=2]
%  #include <stdio.h>
%  #include <unistd.h>
%  #include <sys/types.h>
%  #include <sys/wait.h>
%
%  // This is a comment
%  int main(int argc, char **argv)
%  {
%          while (--c > 1 && !fork());
%          sleep(c = atoi(v[c]));
%          printf("%d\n", c);
%          wait(0);
%          return 0;
%  }
%    \end{lstlisting}}
%    \begin{frame}{Code listings}{An example source code in C}
%      \sleepSort
%    \end{frame}
%
%    \subsection{Citations and Bibliography}
%    \begin{frame}[label=citations]{Citations}
%      \framesubtitle{\TeX, \LaTeX, and Beamer}
%
%      \justifying\TeX\ is a programming language for the typesetting
%      of documents. It was created by Donald Erwin Knuth in the late
%      1970s and it is documented in \emph{The \TeX
%      book}~\cite{knuth84}.
%
%      In the early 1980s, Leslie Lamport created the initial version
%      of \LaTeX, a high-level language on top of \TeX, which is
%      documented in \emph{\LaTeX : A Document Preparation
%      System}~\cite{lamport94}. There exists a healthy ecosystem of
%      packages that extend the base functionality of \LaTeX;
%      \emph{The \LaTeX\ Companion}~\cite{MG94} acts as a guide
%      through the ecosystem.
%
%      In 2003, Till Tantau created the initial version of Beamer, a
%      \LaTeX\ package for the creation of presentations. Beamer is
%      documented in the \emph{User's Guide to the Beamer
%      Class}~\cite{tantau04}.
%    \end{frame}
%

%
%\section{Something else}
%
%\begin{frame}
%\frametitle{There Is No Largest Prime Number}
%\framesubtitle{The proof uses \textit{reductio ad absurdum}.}
%\begin{theorem}
%There is no largest prime number. \end{theorem}
%\begin{enumerate}
%\item<1-| alert@1> Suppose $p$ were the largest prime number.
%\item<2-> Let $q$ be the product of the first $p$ numbers.
%\item<3-> Then $q+1$ is not divisible by any of them.
%\item<1-> But $q + 1$ is greater than $1$, thus divisible by some prime
%number not in the first $p$ numbers.
%\end{enumerate}
%\end{frame}
%
%\begin{frame}{A longer title}
%\begin{itemize}
%\item one
%\item two
%
%\textbf{This is a test of bold text}
%
%\end{itemize}
%\end{frame}
%
%\begin{frame}[allowframebreaks]{Test}
%  First slide
%  \begin{itemize}
%    \item
%    \item
%    \item
%    \item
%    \item
%  \end{itemize}
%  \framebreak
%  Second slide
%  \begin{itemize}
%    \item
%    \item
%    \item
%    \item
%    \item
%  \end{itemize}
%\end{frame}
%%--- Next Frame ---%
%
%
%
\end{document}
